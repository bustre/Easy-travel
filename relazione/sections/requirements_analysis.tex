% !TeX spellcheck = it_IT
\section{Raccolta e analisi dei requisiti}
\subsection{Proprietà del sistema}

Gli \textbf{utenti} che sono registrati nel sistema vengono identificati da uno \underline{username} scelto durante la fase di registrazione, inoltre si memorizzazione i seguenti dati: la password, l'email e la data di iscrizione. Gli utenti vengono classificati in due categorie: i \emph{clienti} e le \emph{agenzie}. Ogni \textbf{cliente} deve fornire le seguenti informazioni: il nome, il cognome, la data di nascita, il sesso e, in modo facoltativo, un numero di telefono. Mentre di ogni \textbf{agenzia} viene riportato: il nome commerciale e la sede legale con l'indirizzo.

Ogni agenzia può offrire diversi \textbf{pacchetti viaggio}. Per ogni pacchetto vengono salvati i seguenti dati: la data di partenza, la data di ritorna, la disponibilità\footnote{Quanti pacchetti sono rimasti.}, il massimo numero di persone che possono partecipare al viaggio\footnote{Esempio: pacchetto famiglia da massimo 4 persone.}, il prezzo di base\footnote{Senza contare il costo dei vari mezzi da prendere per arrivare a destinazione} e un \underline{ID} per identificare univocamente il pacchetto. Ogni pacchetto inoltre riporta le \textbf{informazioni di soggiorno}, dove vengono salvate le informazioni sulle \emph{camere} e sull'\emph{alloggio}. Per ogni \textbf{camera} prenotata vengono riportate: la tipologia\footnote{Esempio: camera da letto, suite, singola, eccetera.} e il codice della camera assegnata. Tra le informazioni di soggiorno ci sono anche i dati dell'\textbf{alloggio}, identificato dal suo \underline{nome} e dalla \underline{città} in cui è ubicato con l'indirizzo, in più viene riportata la tipologia di struttura\footnote{Esempio Hotel, eccetera.} e il numero di stelle se disponibili. Sia il pacchetto di viaggio e l'alloggio hanno una \textbf{descrizione} testuale che viene identificata nel sistema da un \underline{ID}, dove viene riportato: un titolo della descrizione e un testo.

Un cliente può scrivere un \emph{recensione} per l'alloggio alla fine del viaggio. Le \textbf{recensioni} sono identificate da un \underline{ID} interno e riportano: un giudizio (con una scala da 0 a 5) e una motivazione testuale che può essere facoltativa. Un cliente può \emph{prenotare} un pacchetto, nella \textbf{prenotazione} vengono salvati: il numero di persone che partecipano al viaggio, la data di acquisto, i dati della transizione di pagamento e, facoltativamente le informazioni di trasporto per arrivare a destinazione e per il ritorno. Una \textbf{transizione} riporta: un \underline{codice} identificativo, la banca che ha preso in carico l'operazione, l'importo totale, il circuito usato e il timestamp in cui è avvenuta l'operazione. Per ogni prenotazione si può sceglie anche l'offerta più convenite di trasporto. Il \textbf{trasporto} riporta: il prezzo, la data di arrivo, la data di partenza e l'ora di partenza. Vengono inoltre salvati per ogni tratta la città di partenza e la città di arrivo. Per ogni info-trasporto vengono viene riportato anche il \textbf{mezzo utilizzato}, dove viene riportato il codice mezzo e il nome dell'azienda che offre il servizio. I mezzi si dividono in 3 categorie principali: treno dove viene riporta la classe, l'aereo dove vengono riportati: la classe, le info di check-in, la possibilità di avere un baglio in stiva o a mano, in fine c'è il \textbf{taxi} che riporta il numero telefonico del conducente.

Ogni luogo è riconosciuto nel sistema come una \textbf{città} identificata da un codice interno, vengono salvati: il nome e il paese dove si trova la città.

\subsection{Glossario dei termini}
\begin{center}
    \begin{tabularx}{\textwidth}{p{0.16\textwidth} X >{\raggedright\arraybackslash}p{0.13\textwidth} >{\raggedright\arraybackslash}p{0.20\textwidth}}
        \caption{Dizionario termini}\\\toprule\endfirsthead
        \toprule\endhead
        \midrule\multicolumn{4}{r}{\itshape Continua nella pagina successiva}\\\midrule\endfoot
        \bottomrule\endlastfoot
        %
        %
        %
        \textbf{Termine} & \textbf{Descrizione} & \textbf{Sinonimi} & \textbf{Collegamenti} \\
        \midrule
        \textbf{Utente} & Utente generico iscritto al sistema. & &
        \\\midrule
        \textbf{Cliente} & Utente iscritto al servizio. & & Recensione, Prenotazione
        \\\midrule
        \textbf{Agenzia} & Utente iscritto nel sistema che può offrire dei pacchetti viaggio. & & Pacchetto Viaggio, Città
        \\\midrule
        \textbf{Pacchetto} & Soluzione di viaggio offerto da un'agenzia & Pacchetto di viaggio & Prenotazione, Recensione, Informazioni soggiorno, Polizza, Descrizione, Tratta
        \\\midrule
        \textbf{Informazioni Soggiorno} & Informazioni sulle camere e la struttura ospitante per le vacanze & & Pacchetto, Alloggio, Camere
        \\\midrule
        \textbf{Camere} & Camere da letto assegnate al cliente & & Informazioni soggiorno
        \\\midrule
        \textbf{Prenotazione} & Comprare un pacchetto di viaggio & Acquisto, comprare & Cliente, Pacchetto Viaggio
        \\\midrule
        \textbf{Trasporto} & Tratta percorsa da un mezzo di trasporto & & Mezzo Trasporto, Città, Pacchetto di Viaggio
        \\\midrule
        \textbf{Recensione} & Giudizio del cliente testuale e numerico basato su una scala da 0 a 5 & & Cliente, Agenzia, Pacchetto 
        \\\midrule
        \textbf{Descrizione} & Descrizione testuale di un servizio, formata da un titolo e da un testo & & Pacchetto, Alloggio
        \\\midrule
        \textbf{Mezzo} & Mezzo di trasporto che l'agenzia include nel pacchetto per l'andata/ritorno & Mezzo di trasporto, trasporto, servizio di trasporto & Tratta
        \\\midrule
        \textbf{Alloggio} & Dimora per il viaggio & & Città, Informazioni soggiorno, Descrizione
        \\
    \end{tabularx}
\end{center}

\subsection{Operazioni}
Nel caso d'uso perso in esame il numero di operazioni effettuate non hanno una distribuzione uniforme durante tutto l'anno, ma alcune operazioni in particolare presentano un numero di richieste maggiore durante i periodi di vacanza, cioè durante i periodi di \emph{massimo carico} per il sistema, mentre in altri periodi ci sono momenti di \emph{idle}. Riportiamo di seguito le operazioni considerando il massimo delle operazioni registrare per un certo periodo, dato che in contesti come questi la priorità è che il sistema riesca a soddisfare tutte le richieste.
\newline
\begin{center}
    \begin{tabularx}{\textwidth}{L X p{0.22\textwidth}}
        \caption{Tabella delle operazioni}\\\toprule\endfirsthead
        \toprule\endhead
        \midrule\multicolumn{3}{r}{\itshape Continua nella pagina successiva}\\\midrule\endfoot
        \bottomrule\endlastfoot
        %
        %
        \textbf{Operazione} & \textbf{Descrizione dell'operazione} & \textbf{N° operazioni}
        \\
        && (\textbf{operazioni/tempo})\footnote{Riportiamo le misure di tempo: \textbf{dd} = giorni, \textbf{mm} = mesi e \textbf{yy} = anni.}
        \\\midrule
        \emph{Inserimento pacchetto} & Inserimento di un pacchetto da parte di una agenzia & 10 o/dd
        \\\midrule
        \emph{Inserimento descrizione} & Inserimento di una descrizione per un pacchetto o per un luogo di soggiorno & 5 o/dd
        \\\midrule
        \emph{Inserimento mezzo} & Inserimento di un nuovo mezzo per un viaggio & 15 o/mm
        \\\midrule
        \emph{Inserimento trasporto} & Inserimento informazioni e dati riguardanti il servizio di trasporto & 60 o/mm
        \\\midrule
        \emph{Iscrizione utente} & Un utente si iscrive al servizio & 2.000 o/mm
        \\\midrule
        \emph{Inserimento recensione} & Un utente aggiunge una recensione & 2.000 o/mm
        \\\midrule
        \emph{Ricerca pacchetti} & Ricerca dei pacchetti da parte degli utenti & 70.000 o/dd
        \\\midrule
        \emph{Inserimento agenzia} & Viene aggiunta una nuova agenzia & 3 o/mm
        \\\midrule
        \emph{Inserimento città} & Viene aggiunta una nuova città & 10 o/yy
        \\\midrule
        \emph{Acquisto pacchetto} & Un cliente compra un pacchetto & 5.000 o/gg
        \\\midrule
        \emph{Visualizzazione degli acquisti} & Un cliente vuole visualizzare gli acquisti passati & 90 o/gg
        \\
    \end{tabularx}
\end{center}