\section{Raccolta e analisi dei requisiti}
\subsection{Proprietà del sistema}
%
% Incipit
%
Si vuole realizzare una base di dati per un società che permette di confrontare le offerte fornite da diverse agenzie di viaggio, si vuole rappresentare: gli utenti, gli acquisti, le agenzie, i pacchetti, le recensioni, gli alloggi, i mezzi di trasporto, le tratte che percorrono i mezzi e le città.

%
% Utenti
%
Gli \textbf{utenti} sono registrati alla piattaforma, sono identificati da un username scelto durante la fase di registrazione al servizio, inoltre vengono memorizzati per ogni utente:
\begin{multicols}{2}
\begin{itemize}
    \item nome;
    \item cognome;
    \item data di nascita;
    \item luogo di nascita;
    \item password (in sha256);
    \item l'email;
    \item numero di telefono;
    \item data di iscrizione al servizio;
    \item i punti fedeltà.
\end{itemize}
\end{multicols}
%
% Cliente -> Acquisti
%
Gli utenti possono prenotare i \textbf{pacchetti} di viaggio forniti dalle \textbf{agenzie di viaggio} identificate nel database attraverso il LEI (un codice internazionale che identifica univocamente le aziende), inoltre viene salvato il nome dell'agenzia e la sua sede legale. Di ogni acquisto effettuato dall'utente viene salvato:
\begin{multicols}{2}
    \begin{itemize}
        \item data;
        \item numero di minori che partecipano al viaggio;
        \item numero totale (minori compresi) di persone che partecipano al viaggio;
        \item numero di punti fedeltà (usati per l'acquisto e quelli ricevuti da ogni acquisto).
     \end{itemize}
 \end{multicols}
 Un utente riceve dei punti per ogni acquisto, ma non può ricevere punti se li usa durante un acquisto.
%
% Pacchetti
%
I pacchetti vengono identificati da un ID assegnato sempre dal richiedente, i pacchetti hanno:
\begin{multicols}{2}
    \begin{itemize}
        \item prezzo\footnote{Originale, senza nessuno sconto.};
        \item disponibilità (posti disponibili per il viaggio);
        \item data di partenza e ritorno;
        \item numero di persone massime che possono partecipare al viaggio;%
        \footnote{Ovvero un cliente può acquistare un viaggio per 3 persone se e solo se il pacchetto lo prevede.};
        \item la presenza o meno dell'assicurazione per il viaggio.
    \end{itemize}
 \end{multicols}
Inoltre i pacchetti possono avere uno \textbf{sconto} in cambio di un certo numero di punti che un utente possiede.
%
% Trasporto & tratte
%
Ogni pacchetto offre il servizio di \textbf{trasporto} per andare nelle varie destinazioni e per il ritorno. Per arrivare a destinazione o per il ritorno può essere necessario prendere più mezzi, ognuno dei quali percorre una \textbf{tratta}. I mezzi sono identificati attraverso l'azienda che mette a disposizione il veicolo e il codice mezzo utilizzato dall'azienda, essi si dividono in: autobus, aeroplano, treno, taxi e nave; per il treno e l'aeroplano viene riportata anche la classe. Le tratte sono identificate mezzo che percorre la tratta, dall'orario e data di partenza. Ogni tratta tiene le seguente informazioni:
\begin{multicols}{2}
    \begin{itemize}
        \item mezzo utilizzato per la tratta;
        \item orario di partenza;
        \item orario di arrivo;
        \item luogo di partenza;
        \item luogo di arrivo.
    \end{itemize}
\end{multicols}
%
% Destinazioni / descrizione pacchetto & Aloggi
%
Un pacchetto può avere molteplici destinazione (viaggio con più tappe) oppure solo una destinazione (classico), inoltre vengono salvati anche gli \textbf{alloggi} che il turista userà durante il viaggio, per ogni alloggio viene fornita una \textbf{descrizione} e il numero di stelle dell'alloggio se disponibili. Gli alloggi vengono identificati dal nome della struttura e dal luogo in cu si trovano.
%
% Città
%
Le destinazioni di un pacchetto, il luogo di partenza/arrivo di ogni tratta, i luoghi dove si trovano i soggiorni per il viaggio e la sede legale di un'agenzia sono delle \textbf{città}. Una città che viene identificata dal proprio nome e dal paese in cui si trova, inoltre, se è disponibile l'informazione, viene salvata la regione in cui si trova.
%
% Recensioni
%
L'utente può scrivere delle \textbf{recensioni} per il pacchetto e per i luoghi di soggiorno, le recensioni sono identificate ID interno all'azienda. Nella recensione viene salvato un giudizio su una scala da 0 a 5 e una testo che può essere facoltativo.
%
% Descrizione
%
Il pacchetto e l'alloggio ha una \textbf{descrizione}, identificata univocamente da un codice interno all'azienda, dove troviamo: un titolo e una descrizione testuale.
%

\subsection{Glossario dei termini}
\begin{center}
    \begin{tabularx}{\textwidth}{p{0.16\textwidth} X p{0.15\textwidth} p{0.18\textwidth}}
        \caption{Dizionario termini}\\\toprule\endfirsthead
        \toprule\endhead
        \midrule\multicolumn{4}{r}{\itshape Continua nella pagina successiva}\\\midrule\endfoot
        \bottomrule\endlastfoot
        %
        %
        %
        \textbf{Termine} & \textbf{Descrizione} & \textbf{Sinonimi} & \textbf{Collegamenti} \\
        \midrule
        \textbf{Utente} & Utente iscritto al servizio. & Cliente & Pacchetti, Recensioni, Prenota
        \\\midrule
        \textbf{Pacchetto} & Soluzione di viaggio offerto da un'agenzia di viaggio & Pacchetto di viaggio & Cliente, Recensioni, Città, Soggiorno, Tratte, Prenota
        \\\midrule
        \textbf{Prenota-zione} & Comprare un pacchetto di viaggio & Acquisto, comprare & Cliente, Pacchetto
        \\\midrule
        \textbf{Tratta} & Tratta percorsa da un mezzo di trasporto & & Mezzo, Città, Pacchetto
        \\\midrule
        \textbf{Agenzia di viaggio} & Agenzia che offre i pacchetti di viaggio & Agenzia & Pacchetto, Città
        \\\midrule
        \textbf{Richiedente} & Compagnia che offre il servizio e che ha richiesto lo sviluppo della database & &
        \\\midrule
        \textbf{Recensione} & Giudizio del cliente testuale e numerico basato su una scala di valori & & Soggiorno, Pacchetto %TODO: Descrizione merdosa
        \\\midrule
        \textbf{Descrizione} & Descrizione testuale di un servizio, formata da un titolo e da un testo & & Pacchetto, Soggiorno
        \\\midrule
        \textbf{Mezzo} & Mezzo di trasporto che l'agenzia include nel pacchetto per l'andata/ritorno & Mezzo di trasporto, trasporto, servizio di trasporto & Tratta
        \\\midrule
        \textbf{Alloggio} & Dimora temporanea per il viaggio & & Città, Pacchetto, Descrizione, Recensione
        \\\midrule
        \textbf{Sconto} & Sconto applicabile al prezzo originale del pacchetto, può presentare dei requisti per essere applicato & & Pacchetto
        \\
    \end{tabularx}
\end{center}

\subsection{Operazioni}
Nel caso d'uso perso in esame il numero di operazioni effettuate non hanno una distribuzione uniforme durante l'anno, ma alcune operazioni in particolare presentano un numero di richieste maggiore durante i periodi di vacanza, cioè durante i periodi di \emph{massimo carico} per il sistema, mentre in altri periodi ci sono momenti di \emph{idle}. Riportiamo di seguito le operazioni considerando il massimo di operazioni registrare per un certo periodo, dato che in contesti come questi la priorità è che il sistema riesca a soddisfare tutte le richieste.
\newline
\begin{center}
    \begin{tabularx}{\textwidth}{p{0.2\textwidth} X p{0.25\textwidth}}
        \caption{Tabella delle operazioni}\\\toprule\endfirsthead
        \toprule\endhead
        \midrule\multicolumn{3}{r}{\itshape Continua nella pagina successiva}\\\midrule\endfoot
        \bottomrule\endlastfoot
        %
        %
        \textbf{Operazione} & \textbf{Descrizione dell'operazione} & \textbf{N°. operazioni}
        \\
        && (\textbf{operazioni/tempo})\footnote{Riportiamo le misure di tempo: \textbf{dd} = giorni, \textbf{mm} = mesi \& \textbf{yy} = anni.}
        \\\midrule
        \emph{Inserimento pacchetto} & Inserimento di un pacchetto da parte di una agenzia & 10 o/dd
        \\\midrule
        \emph{Inserimento descrizione} & Inserimento di una descrizione per un pacchetto o per un luogo di soggiorno & 15 o/dd
        \\\midrule
        \emph{Inserimento mezzo} & Inserimento nuovo mezzo per un viaggio & 2 o/mm
        \\\midrule
        \emph{Iscrizione utente} & Un utente si iscrive al servizio & 40 o/dd
        \\\midrule
        \emph{Inserimento recensione} & Un utente aggiunge una recensione & 3.000 o/mm
        \\\midrule
        \emph{Ricerca pacchetti} & Ricerca dei pacchetti da parte degli utenti & 600.000 o/dd
        \\\midrule
        \emph{Inserimento agenzia} & Viene aggiunta una nuova agenzia & 3 o/mm
        \\\midrule
        \emph{Inserimento città} & Viene aggiunta una nuova città & 10 o/yy
        \\\midrule
        \emph{Inserimento sconto} & Inserimento di un nuovo sconto & 8 o/dd
        \\\midrule
        \emph{Acquisto pacchetto} & Un cliente acquista un pacchetto & 100.000 o/gg
        \\\midrule
        \emph{Visualizzazione degli acquisti} & Un cliente vuole visualizzare gli acquisti passati & 90 o/gg
        \\\midrule
        \emph{Visualizzazione dei punti} & Un cliente vuole vedere i punti a sua disposizione & 1.600 o/gg
        \\
    \end{tabularx}
\end{center}